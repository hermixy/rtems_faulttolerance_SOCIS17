% introduction.tex
\chapter{Introduction}
\section{Motivation}
Mobile and embedded systems are susceptible to transient faults in the underlying hardware~\cite{1545891}, which may occur due to rising integration density, low voltage operation and environmental influences such as radiation and electromagnetic interference. Transient faults may alter the execution state or incur soft-errors. Bitflips are a direct cause of transient faults, and occur in register, processor, main memory, and other parts of a system.
The consequences of bitflips are difficult to predict, but in the worst case they may lead to catastrophic events such as unrecoverable system failure. Recently, a Japanese satellite Hitomi crashed, because its control loop got corrupted. According to investigations in ~\cite{hitomi} a bitflip occurred in the satellite's rotation control, which made it rotate uncontrollably; it broke into several pieces. The financial damage was severe; a few hundred million Euros.
%--
To prevent such catastrophes, one way is to utilize software-based approaches such as redundant execution and error-correction code~\cite{1395730, 994913, 6513800, 4700445, 1137644}. %TODO add avionics 9 times example
Yet, trivially adopting redundant execution or error
correction code may lead to significant computational  overhead, e.g., when N+1 redundancy is used. Due  to  spatial
limitation  and  timeliness,  skillfully  adopting  software-based
fault-tolerance  approaches  to  prevent system  failure  due  to  transient
faults is not a trivial problem.

Instead of over-provisioning with extra hardware or execution time, sometimes errors can be tolerated; there may be no need to protect the whole system. Due to the potential inherent
safety  margins  and  noise  interference,  control  applications
might  tolerate  a  limited  number  of  errors  with  a  downgrade
of  control  performance  without  leading  to  an  unrecoverable
system state. In previous studies, techniques have been proposed for delayed~\cite{Ramanathan:1999:OMR:307977.307993, 6241580} or dropped~\cite{4739306,7153812} signal samples. Chen et al. \cite{Chen2016}  explored  the  fault  tolerance  of  a
LegoNXT  path-tracing  control  application.  To  prevent the  system  from mission  failures, they  proposed  to use the $(m,k)$-firm real-time task model to quantify the robustness of control tasks, which means that $m$ out of $k$
consecutive task  instances  need  to  be  correct, called $(m,k)$
robustness  requirement,  using which they provided compensation techniques to manage the expensive error correction to avoid overprovision. 

\section{Project Goal}

The purpose of this project is to provide an implementation of the techniques in~\cite{Chen2016} on the real-time operating system RTEMS \cite{rtemsweb}. In the previous study, the application solely runs on a LegoNXT robot using nxtOSEK, and is not portable. An implementation in RTEMS allows us to port the application to several other platforms; we can test the techniques on  other hardware, e.g., Raspberry Pi, to reach more general conclusions about fault tolerance of control tasks and the proposed soft-error handling techniques. The application including the fault tolerance techniques is an option for space vehicles to manage redundant executions in the future.   

\section{Report Structure}
The background of transient faults and the type of faults the Fault Tolerant Scheduler (FTS) can protect from is presented in the second chapter.
The system model and notation is introduced in Chapter 3, including the task versions, the $(m,k)$ robustness requirement, and the schedulability and scheduling. 
The soft-error handling techniques S-RE, S-DR, D-RE, and D-DR, which decide when to execute which task version, are the central theme in Chapter 4. Their theoretical background is covered briefly as well.
In Chapter 5, the integration of the FTS in RTEMS, and tutorials on how to add and use the FTS are presented.
